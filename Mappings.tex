The project has $2$ ways to denote position, using the array element index or position $x$. The functions \url{x2i} and \url{i2x} provides an easy way to switch between these $2$ coordinate notations.
\\~\\The function \url{Equate} is somewhat of a relic from the previous project. This function is a way to copy array elements from one array to another. This is important when marching time steps or setting some of the placeholder arrays \url{rho}, \url{Ve} from one of the solution arrays.
\\~\\The next part of the utilities which are important are \url{FluidDensity}, \url{FluidVelocity}, \url{FluidPressure}, and \url{FlSpeeda}. These functions provide a way to use either the old or new solution arrays and obtain the primitive variables. The primitive variables can then be used for computations for the case of fluxes, or can be used for printing. It is imporatnt to note that the functions \url{FluidDensity}, \url{FluidVelocity}, \url{FluidPressure}, and \url{FlSpeeda} are all dependent on each other, so they must be run sequentially for the primitive variable arrays to be computed correctly.
