The time iteration loop is shown below in the function \url{Simulate}. The density, velocity, pressure and speed of sound arrays \url{rho}, \url{Ve}, \url{Pr}, \url{a_s} are first updated using the old time step solution. This means at the beginning of every time iteration, the placeholder arrays \url{rho}, \url{Ve}, \url{Pr} and \url{a_s} represent the last time step. 
\\~\\After that, then boundary conditions are applied. After Boundary conditions are applied, the solution of the old time step is copied over to the new time step. This is to represent the first $LHS$ term in equation $\ref{Explicit Formulation Primitive}$.
\\~\\After that has been done, then the positive fluxes are computed and applied to the solution array, which accounts for the positive fluxes in equation $\ref{Explicit Formulation Primitive}$. After that is done, the negative fluxes are computed and then applied to the solution array which accounts for the negative fluxes in the equation $\ref{Explicit Formulation Primitive}$. The arrays \url{Flux1}, \url{Flux2}, and \url{Flux3}, are used to facilitate the communication of fluxes between the flux computing function and the flux application functions.
\\~\\After that, it is assumed that equation $\ref{Explicit Formulation Primitive}$ is implemented fully and so it is acceptable to move to the next time step. The placeholder arrays for density, velocity, and pressure are printed after all computations are done for the current time step. The way to move to the next time step is by copying the new solution arrays to the old solution arrays, and this is done as the last step in any given time step.
